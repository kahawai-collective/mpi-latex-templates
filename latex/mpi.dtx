% \iffalse meta-comment
% The MIT License (MIT)
%
% Copyright (c) 2014 Dragonfly Science
%
% Permission is hereby granted, free of charge, to any person obtaining a copy
% of this software and associated documentation files (the "Software"), to deal
% in the Software without restriction, including without limitation the rights
% to use, copy, modify, merge, publish, distribute, sublicense, and/or sell
% copies of the Software, and to permit persons to whom the Software is
% furnished to do so, subject to the following conditions:
%
% The above copyright notice and this permission notice shall be included in all
% copies or substantial portions of the Software.
%
% THE SOFTWARE IS PROVIDED "AS IS", WITHOUT WARRANTY OF ANY KIND, EXPRESS OR
% IMPLIED, INCLUDING BUT NOT LIMITED TO THE WARRANTIES OF MERCHANTABILITY,
% FITNESS FOR A PARTICULAR PURPOSE AND NONINFRINGEMENT. IN NO EVENT SHALL THE
% AUTHORS OR COPYRIGHT HOLDERS BE LIABLE FOR ANY CLAIM, DAMAGES OR OTHER
% LIABILITY, WHETHER IN AN ACTION OF CONTRACT, TORT OR OTHERWISE, ARISING FROM,
% OUT OF OR IN CONNECTION WITH THE SOFTWARE OR THE USE OR OTHER DEALINGS IN THE
% SOFTWARE.
% \fi
%
% \iffalse
%<common|mpi-aebr|mpi-far|mpi-plenary>\NeedsTeXFormat{LaTeX2e}[1999/12/01]
%
%<common>\ProvidesPackage{mpi}[2015/01/09 v0.1 Common formatting requirements for MPI reports]
%<mpi-aebr>\ProvidesClass{mpi-aebr}[2015/01/09 v0.1 MPI AEBR format]
%<mpi-far>\ProvidesClass{mpi-far}[2015/01/09 v0.1 MPI FAR format]
%<mpi-plenary>\ProvidesClass{mpi-plenary}[2015/01/09 v0.1 MPI FAR format]
%<*driver>
\documentclass{ltxdoc}
\usepackage{mpi}
\usepackage[toc]{multitoc}

\usepackage{tocloft}
\setcounter{tocdepth}{2}

\EnableCrossrefs
\CodelineIndex
\RecordChanges
\begin{document}
  \DocInput{mpi.dtx}
\end{document}
%</driver>
%\fi
%
% \CheckSum{0}
%
% \CharacterTable
%  {Upper-case    \A\B\C\D\E\F\G\H\I\J\K\L\M\N\O\P\Q\R\S\T\U\V\W\X\Y\Z
%   Lower-case    \a\b\c\d\e\f\g\h\i\j\k\l\m\n\o\p\q\r\s\t\u\v\w\x\y\z
%   Digits        \0\1\2\3\4\5\6\7\8\9
%   Exclamation   \!     Double quote  \"     Hash (number) \#
%   Dollar        \$     Percent       \%     Ampersand     \&
%   Acute accent  \'     Left paren    \(     Right paren   \)
%   Asterisk      \*     Plus          \+     Comma         \,
%   Minus         \-     Point         \.     Solidus       \/
%   Colon         \:     Semicolon     \;     Less than     \<
%   Equals        \=     Greater than  \>     Question mark \?
%   Commercial at \@     Left bracket  \[     Backslash     \\
%   Right bracket \]     Circumflex    \^     Underscore    \_
%   Grave accent  \`     Left brace    \{     Vertical bar  \|
%   Right brace   \}     Tilde         \~}
%
% \changes{0.2}{2015/02/01}{Initial Version + fixes}
%
% \GetFileInfo{mpi.dtx}
%
% \title{Formatting Fisheries New Zealand reports}
% \author{Dragonfly Science}
% \maketitle
%
% \begin{abstract}
%  This is a set of classes that are used to format \LaTeX{} documents according to the
%  Fisheries New Zealand formatting requirements. There are three document
%  classes provided: ``mpi-aebr'', for formatting Aquatic Environment and Biodiversity
%  Reports, ``mpi-far'', for formatting Fisheries Assessment Reports and ``mpi-plenary``
%  for stock assessment Plenary Documents. Using these packages allows the production of 
%  publication ready documents.
% \end{abstract}
% \clearpage
% \tableofcontents
%  \clearpage
% \section{Class and Package Options}
%
% All options are passed down to the common layer.
%
% Since we intend on manually hiding certain options we need a command to
% clear it from the global options list. This is taken from
% \url{http://tex.stackexchange.com/questions/33245/disabling-the-draft-option-in-a-package}.
% The package needs to recognise the draft option, and just
% give it to ifdraft. The rest are just default behaviour.
%
%    \begin{macrocode}
%<*common>
\def\@clearglobaloption#1{%
  \def\@tempa{#1}%
  \def\@tempb{\@gobble}%
  \@for\next:=\@classoptionslist\do
    {\ifx\next\@tempa
       \message{Cleared  option \next\space from global list}%
     \else
       \edef\@tempb{\@tempb,\next}%
     \fi}%
  \let\@classoptionslist\@tempb
  \expandafter\ifx\@tempb\@gobble
    \let\@classoptionslist\@empty
  \fi}

\DeclareOption{draft}{
  \PassOptionsToPackage{\CurrentOption}{ifdraft}
}
\ProcessOptions\relax

\DeclareOption{draft}{
  \PassOptionsToPackage{\CurrentOption}{mpi}
  \@clearglobaloption{draft}
}
\ProcessOptions\relax
%</common>
%    \end{macrocode}
%
% \section{Templates}
% Each different class is based on a different template. The following
% styles have classes implemented:
%
% \begin{itemize}
% \item Aquatic Environment and Biodiversity Report
% \item Fisheries Assessment Report
% \item Fisheries Assessment Plenary
% \end{itemize}
%
% The basic structure is the same as the base latex article class
% with small changes to the theme. All of the classes
% use a commmon style to share general definitions.
% This is provided by the mpi style.
%
%    \begin{macrocode}
%<mpi-aebr|mpi-far|mpi-plenary>\LoadClass[a4paper,twoside,final,11pt]{article}
%<mpi-aebr|mpi-far|mpi-plenary>\RequirePackage{mpi}
%    \end{macrocode}
%
% \section{Page Layout}
%
% The MPI styles use margins based on existing reports.
%
%    \begin{macrocode}
%<common>\RequirePackage[margin=2.5cm]{geometry}
%    \end{macrocode}
%
% \section{Fonts}
%
% MPI reports use standard times like fonts.
%
%    \begin{macrocode}
%<*mpi-aebr|mpi-far|mpi-plenary>
\RequirePackage{amsmath}
\RequirePackage{mathspec}
\setallmainfonts{Times New Roman}
\setsansfont[Mapping=tex-text,
  ItalicFont = {Arial Italic},
  BoldFont = {Arial Bold},
  BoldItalicFont = {Arial Bold Italic}]{Arial}
\setmathsf[Mapping=tex-text,
    Numbers=Monospaced,
    ItalicFont     = {Times New Roman Italic},
    BoldFont       = {Times New Roman Bold},
    BoldItalicFont = {Times New Roman Bold Italic}]{Times New Roman}
\setmathtt[Mapping=tex-text,
    Numbers=Monospaced,
    ItalicFont     = {Times New Roman Italic},
    BoldFont       = {Times New Roman Bold},
    BoldItalicFont = {Times New Roman Bold Italic}]{Times New Roman}
%</mpi-aebr|mpi-far|mpi-plenary>
%    \end{macrocode}
%
% We also set the fontsize to be 11pt.
%    \begin{macrocode}
%<common>\renewcommand{\normalsize}{\fontsize{11pt}{13pt}\selectfont}
%    \end{macrocode}
%% Paragraphs are not indented, but there is extra spacing between them.
%    \begin{macrocode}
%<common>\setlength{\parskip}{11pt}
%<common>\setlength{\parindent}{0mm}
%    \end{macrocode}
%
% Hyphenation is discouraged.
%    \begin{macrocode}
%<common>\tolerance=1
%<common>\emergencystretch=\maxdimen
%<common>\hyphenpenalty=10000
%<common>\hbadness=10000
%    \end{macrocode}
%
% \section{Headings}
%    \begin{macrocode}
%<*mpi-aebr|mpi-far|mpi-plenary>
\RequirePackage[nobottomtitles]{titlesec}
\titleformat{\section}{\normalfont\bfseries\sffamily\uppercase}{}{0em}{\thesection.\hspace{0.8em}}
\titleformat{name=\section,numberless}{\normalfont\bfseries\sffamily\uppercase}{}{0em}{}
\titleformat{\subsection}{\normalfont\bfseries\sffamily}{}{0em}{\thesubsection\hspace{0.8em}}
\titleformat{\subsubsection}{\normalfont\bfseries\sffamily}{}{0em}{\thesubsubsection\hspace{0.8em}}
\titlespacing{\section}{0pt}{\baselineskip}{0pt}
\titlespacing{\subsection}{0pt}{\baselineskip}{0pt}
\titlespacing{\subsubsection}{0pt}{\baselineskip}{0pt}
%</mpi-aebr|mpi-far|mpi-plenary>
%    \end{macrocode}
%
% \section{Header and Footer}
%
%    Page footers must contain the report name, page and publisher.
%
%    \begin{macrocode}
%<*mpi-aebr|mpi-far|mpi-plenary>
\usepackage{fancyhdr}

\fancypagestyle{plain}{
    \fancyhf{}
    \fancyfoot[LO,RE]{\sffamily\scriptsize Fisheries New Zealand}
    \fancyfoot[LE]{\sffamily\scriptsize\thepage\ {\normalsize$\bullet$}\ \footertitle}
    \fancyfoot[RO]{\sffamily\scriptsize\footertitle\ {\normalsize$\bullet$}\ \thepage}
    \renewcommand{\footrulewidth}{0.3pt}
    \renewcommand{\headrulewidth}{0pt}
  }
%</mpi-aebr|mpi-far|mpi-plenary>
%    \end{macrocode}
%
% \section{Bibliography}
%
% Since we are using biblatex with apa we need to use polyglossia to tell
% biblatex which language we are using. Actually polyglossia doesn't support
% british directly so we need a slight hack. We also need the bibliography to be
% a proper section. We also define some helper cite commands for familiarity.
%
%    \begin{macrocode}
%<*mpi-aebr|mpi-far|mpi-plenary>
\RequirePackage{csquotes}
\RequirePackage{polyglossia}
\RequirePackage{xpatch}
\setdefaultlanguage[variant=newzealand]{english}
\RequirePackage[style=apa,
    backend=biber,
    uniquename=false,
    uniquelist=false,
    sorting=ynt, % Override apa sort; switch back below for the bibliography
    apamaxprtauth=999,
    datamodel=affiliation]{biblatex}
\DeclareLanguageMapping{english}{english-apa}
\defbibheading{bibliography}[References]{\section{#1}}
\newcommand{\citet}{\textcite}
\newcommand{\citep}{\parencite}
\renewcommand{\cite}{\textcite}
%</mpi-aebr|mpi-far|mpi-plenary>
%    \end{macrocode}
%
% Newer versions of biblatex have better support for different sort keys
% on citations and bibliography. Until then, override \\printbibliography
%
%    \begin{macrocode}
%<*mpi-aebr|mpi-far|mpi-plenary>
\let\oldprintbibliography\printbibliography
\renewcommand*\printbibliography{
  \endrefcontext
  \begin{refcontext}[sorting=apa]
    \oldprintbibliography
  \end{refcontext}
}
%</mpi-aebr|mpi-far|mpi-plenary>
%    \end{macrocode}
%
% Use a comma between multiple citations:
%    \begin{macrocode}
%<mpi-aebr|mpi-far|mpi-plenary>\renewcommand*{\multicitedelim}{,\addspace}
%    \end{macrocode}
%
% Use a semicolon between multiple names in text citations:
%    \begin{macrocode}
%<*mpi-aebr|mpi-far|mpi-plenary>
\renewcommand*{\multinamedelim}{\addcomma\addspace}
\renewcommand*{\finalnamedelim}{\addspace\&\addspace}
%</mpi-aebr|mpi-far|mpi-plenary>
%    \end{macrocode}
%
% in bibliography:
%    \begin{macrocode}
%<*mpi-aebr|mpi-far|mpi-plenary>
\DeclareDelimFormat[bib,biblist]{finalnamedelim}{\addsemicolon\addspace}
\AtBeginBibliography{%
  \renewcommand*{\bibinitdelim}{}%
  \renewcommand*{\revsdnamedelim}{}%
  \renewcommand*{\bibpagespunct}{\addcolon\addspace}%
}
%</mpi-aebr|mpi-far|mpi-plenary>
%    \end{macrocode}
%
% [DISABLED] Have a maximum of two authors in citation commands:
%    \begin{macrocode}
%<*mpi-aebr|mpi-far|mpi-plenary>
% Patches https://github.com/plk/biblatex-apa/blob/v6.8/tex/latex/biblatex-apa/cbx/apa.cbx
% \xpatchnameformat{labelname}{\ifciteseen}{\ifnumcomp{\value{listtotal}}{>}{2}}{}{Patch of labelname failed}
%</mpi-aebr|mpi-far|mpi-plenary>
%    \end{macrocode}
%
% Redefine the nameyeardelim command, to put a space between the authors and the year in citations
% in citations, updates https://github.com/plk/biblatex-apa/blob/v6.8/tex/latex/biblatex-apa/cbx/apa.cbx
%    \begin{macrocode}
%<*mpi-aebr|mpi-far|mpi-plenary>
\DeclareDelimFormat[bib]{nameyeardelim}{\space\nopunct}
\renewcommand*{\nameyeardelim}{\space}
%</mpi-aebr|mpi-far|mpi-plenary>
%    \end{macrocode}
%
% [DISABLED] Update the author macro so that there is no period after the authors, for authors without initials, in the bibliography
% Patches https://github.com/plk/biblatex-apa/blob/v6.8/tex/latex/biblatex-apa/bbx/apa.bbx
%    \begin{macrocode}
%<*mpi-aebr|mpi-far|mpi-plenary>
% \xpatchbibmacro{author}{\newunit\newblock}{}{}{Patch of author macro failed}
%</mpi-aebr|mpi-far|mpi-plenary>
%    \end{macrocode}
%
% Turn extra commas inserted by name:apa:family-given into semicolons
%    \begin{macrocode}
%<mpi-aebr|mpi-far|mpi-plenary>\xpatchbibmacro{name:apa:family-given}{\apablx@ifrevnameappcomma{\addcomma}}{\apablx@ifrevnameappcomma{\addsemicolon}}{}{Patch of name:apa:family-given macro failed}
%    \end{macrocode}
%
% Not sure what this one is doing!
%    \begin{macrocode}
%<*mpi-aebr|mpi-far|mpi-plenary>
\xpatchbibmacro{date+extradate}{%
  \printtext[parens]%
  }{%
    \setunit{\addperiod\space}%
      \printtext%
      }{}{}
%</mpi-aebr|mpi-far|mpi-plenary>
%    \end{macrocode}
%
% Add location+publisher to book
%    \begin{macrocode}
%<mpi-aebr|mpi-far|mpi-plenary>\xpatchbibdriver{book}{\printlist{publisher}}{\usebibmacro{location+publisher}}{}{Patch of book driver failed}
%    \end{macrocode}
%
% Change the order of the publisher and the location, so it is `publisher, location` rather than `location: publisher`, in the bibliography:
%    \begin{macrocode}
%<*mpi-aebr|mpi-far|mpi-plenary>
\xpatchbibmacro{location+publisher}{%
  \printlist{location}%
  \setunit*{\addcomma\space}%
  \printlist{publisher}}{%
  \printlist{publisher}%
  \setunit*{\addcomma\space}%
  \printlist{location}}{}{Patch of location+publisher macro failed}
%</mpi-aebr|mpi-far|mpi-plenary>
%    \end{macrocode}
%
% Do not italisise the title of books or collections in the bibliography:
%    \begin{macrocode}
%<*mpi-aebr|mpi-far|mpi-plenary>
\DeclareFieldFormat{booktitle}{#1} %Overriding definition in biblatex.def
\DeclareFieldFormat{maintitle}{#1} %Overriding definition in biblatex.def
\DeclareFieldFormat[thesis]{title}{#1}
\DeclareFieldFormat[unpublished]{title}{#1}
%</mpi-aebr|mpi-far|mpi-plenary>
%    \end{macrocode}
%
% Italicise the "In:" in collections in the bibliography:
%    \begin{macrocode}
%<mpi-aebr|mpi-far|mpi-plenary>\xpatchbibmacro{in}{\bibcpstring{in}\setunit{\space}}{\mkbibemph{\bibcpstring{in}}\addcolon\setunit{\space}}{}{Patch of 'in' macro failed}
%    \end{macrocode}
%
% No comma after the journal title in the bibliography:
%    \begin{macrocode}
%<mpi-aebr|mpi-far|mpi-plenary>\xpatchbibmacro{journal+issuetitle}{\addcomma}{}{}{}
%    \end{macrocode}
%
% Italicise the number in articles
%    \begin{macrocode}
%<mpi-aebr|mpi-far|mpi-plenary>\DeclareFieldFormat[article]{number}{\mkbibemph{\mkbibparens{\apanum{#1}}}}
%    \end{macrocode}
%
% Space between the volume and the number in articles
%    \begin{macrocode}
%<mpi-aebr|mpi-far|mpi-plenary>\xpatchbibmacro{journal+issuetitle}{\printfield{number}}{\setunit{\addspace}\printfield{number}}{}{}
%    \end{macrocode}
%
% Always finish with a period
%    \begin{macrocode}
%<mpi-aebr|mpi-far|mpi-plenary>\renewbibmacro*{apa:finpunct}{\addperiod} % finpunct is gone
%    \end{macrocode}
%
% Add full stop after title for theses
%    \begin{macrocode}
%<mpi-aebr|mpi-far|mpi-plenary>\DeclareFieldFormat[thesis]{title}{#1\addperiod}
%    \end{macrocode}
%
% Gets rid of parentheses around page number
%    \begin{macrocode}
%<mpi-aebr|mpi-far|mpi-plenary>\DeclareFieldFormat{addendum}{{#1}}
%    \end{macrocode}
%
% \section{Title page}
%
% Each kind of document has a different kind of header.
% Each one shall be addressed separately. Additionally,
% each title page section for a class will be split into two parts:
% the first defines new commands for entering metadata into the titlepage
% and the second defines the actual creation of the title page.
%
% Since some of the headings require absolute positioning,
% we load the textpos package, and a number of other packages to help with the
% title pages. We also set the graphics path so we can find common images.
%
%    \begin{macrocode}
%<*common>
\RequirePackage[absolute,overlay]{textpos}
\RequirePackage{etoolbox}
\RequirePackage{xparse}
\RequirePackage{hyphenat}
\RequirePackage{graphicx}
\graphicspath{{./}{/usr/share/texlive/texmf-dist/tex/latex/mpi/}}
%</common>
%    \end{macrocode}

% \subsubsection{Meta Data}
%
% The document has two titles. The first is the main title which goes on
% the cover of the document, while the second is a short title that goes
% in the page footer. Both are required.
%
%    \begin{macrocode}
%<*mpi-aebr|mpi-far|mpi-plenary>
\renewcommand\title[2]{%
  \renewcommand\@title{#2}%
  \renewcommand\footertitle{#1}%
}
\newcommand{\footertitle}{\ClassError{mpi}{No Footer title defined}{Use \noexpand\title}}
%</mpi-aebr|mpi-far|mpi-plenary>
%    \end{macrocode}
%
%
% Like in a standard document |\date| is allowed. However, it is only used for
% creating citations, and as such should contain only the year. The
% default is changed from |\today| to the current year. There
% is also a reportmonth command to set the month.
%
%    \begin{macrocode}
%<*mpi-aebr|mpi-far|mpi-plenary>
\renewcommand{\@date}{\the\year}
\newcommand{\reportmonth}[1]{\renewcommand{\@month}{#1}}
\newcommand{\@month}{MMM}
%</mpi-aebr|mpi-far|mpi-plenary>
%    \end{macrocode}
%
%
% AEBR and FAR documents require ISSN, ISBN and report numbers. All of these must have
% sensible defaults before they are finished. For AEBRs the report number has a prefix No.
%    \begin{macrocode}
%<*mpi-aebr>
\newcommand{\@reportwallpaper}{AEBR_cover.pdf}
\newcommand{\@issn}{1179-6480}
\newcommand{\@reportseries}{New Zealand Aquatic Environment and Biodiversity Report}
\newcommand{\@reportnopre}{No.\ }
%</mpi-aebr>
%<*mpi-far>
\newcommand{\@reportwallpaper}{FAR_cover.pdf}
\newcommand{\@issn}{1179-5352}
\newcommand{\@reportseries}{New Zealand Fisheries Assessment Report}
\newcommand{\@reportnopre}{}
%</mpi-far>
%<*mpi-plenary>
\newcommand{\@reportwallpaper}{PLENARY.jpg}
\newcommand{\@issn}{1179-5352}
\newcommand{\@reportseries}{}
\newcommand{\@reportnopre}{}
%</mpi-plenary>
%<*mpi-aebr|mpi-far|mpi-plenary>
\newcommand{\issn}[1]{\renewcommand{\@issn}{#1}}
\newcommand{\isbn}[1]{\renewcommand{\@isbn}{#1}}
\newcommand{\@isbn}{XX-XXXXX-XX}
\newcommand{\@reportno}{??}
\newcommand{\reportno}[1]{\renewcommand{\@reportno}{#1}}
\newcommand{\reportseries}[1]{\renewcommand{\@reportseries}{#1}}
\newcommand{\reportnopre}[1]{\renewcommand{\@reportnopre}{#1}}
\newcommand{\reportwallpaper}[1]{\renewcommand{\@reportpaper}{#1}}
%</mpi-aebr|mpi-far|mpi-plenary>
%<*mpi-aebr|mpi-far>
\newcommand{\@plainlanguagesummary}{}
\newcommand{\plainlanguagesummary}[1]{\renewcommand{\@plainlanguagesummary}{#1}}
%</mpi-aebr|mpi-far>
%    \end{macrocode}
%
% The report also has a customisable citation note that goes at the end of the
% citation. This defaults to the subtitle.
%
%
% \subsubsection{Maketitle}
%
% We use the wallpaper package to support background images, and
% redefine the maketitle command to generate the right content.
%    \begin{macrocode}
%<common>\RequirePackage{wallpaper}
%<mpi-aebr|mpi-far|mpi-plenary>\renewcommand{\maketitle}{
%    \end{macrocode}
%
% We start by generating metadata for the pdf. We need to at least set the pdf
% author and title.
%    \begin{macrocode}
%<*mpi-aebr|mpi-far|mpi-plenary>
  \frenchspacing
  {
    \renewcommand{\and}{ and }
    \hypersetup{
      pdfinfo={
        Title={\footertitle},
        Author={\@author}
      }
    }
  }
%</mpi-aebr|mpi-far|mpi-plenary>
%    \end{macrocode}
%
% The first page is fully covered by the special title image used by AEBRs.
%
%    \begin{macrocode}
%<*mpi-aebr|mpi-far|mpi-plenary>
\pagestyle{empty}
\clearpage
\ThisCenterWallPaper{1.0}{\@reportwallpaper}
%</mpi-aebr|mpi-far|mpi-plenary>
%    \end{macrocode}
%
% FNZ and NZ government logos
%
%    \begin{macrocode}
%<common>\RequirePackage{tikz}
%<*mpi-aebr|mpi-far>
\begin{tikzpicture}[remember picture, overlay]
    \node [shift={(83mm, -21.5mm)}] at (current page.north west)
        { \includegraphics[scale=0.20]{Fisheries New Zealand - Logo - blue & black -2.jpg} };
\end{tikzpicture}
\begin{tikzpicture}[remember picture, overlay]
    \node [shift={(-48mm, 13mm)}] at (current page.south east)
        { \includegraphics[scale=0.16]{NZGovt-logo-expanded-wordmark.png} };
\end{tikzpicture}
%</mpi-aebr|mpi-far>
%    \end{macrocode}
%
% The text is overlayed into a small box over the image, such that is it over the
% white part of the image.
%
%    \begin{macrocode}
%<*mpi-aebr|mpi-far|mpi-plenary>
\begin{textblock*}{0.7\textwidth}(86.1mm,61.5mm)
\begin{minipage}[t][77mm]{1.0\linewidth}
%</mpi-aebr|mpi-far|mpi-plenary>
%    \end{macrocode}
%
% The entire titlepage is written in sans serif font.
%
%    \begin{macrocode}
%<*mpi-far|mpi-plenary>
  {\sffamily
%</mpi-far|mpi-plenary>
%<*mpi-aebr>
  {\fontspec{Arial Narrow}
%</mpi-aebr>
%    \end{macrocode}
%
% The title is written in large bold font. It is ragged right rather than justified
% and should avoid hyphenation.
%
%    \begin{macrocode}
%<*mpi-aebr|mpi-far|mpi-plenary>
  {
    \raggedright
    \fontsize{18pt}{22pt}
    \bfseries
    \selectfont
    \nohyphens{\@title}\par
    \vfill
  }

%</mpi-aebr|mpi-far|mpi-plenary>
%    \end{macrocode}
%
% Each report then states that is an AEBR and its number.
%
%    \begin{macrocode}
%<*mpi-aebr|mpi-far>
  {\@reportseries} \@reportnopre {\@reportno}
  \vfill
%</mpi-aebr|mpi-far>
%    \end{macrocode}
%
% The authors are printed in a slightly indented list.
%
%    \begin{macrocode}
%<*mpi-aebr|mpi-far|mpi-plenary>
\DeclareNameFormat{titlename}{\usebibmacro{name:given-family}%
   {\namepartfamily}%
   {\namepartgiveni}%
   {\namepartprefix}%
   {\namepartsuffix}}
\newcommand{\getauthors}{
  \begin{refsection}[\jobname-self.bib]
  \renewrobustcmd*{\bibnamedelimi}{}
  \renewcommand*{\multinamedelim}{,\\}
  \renewcommand*{\finalnamedelim}{,\\}%
  \AtNextCite{\defcounter{maxnames}{99}}
  \citename{this}[titlename]{author}
  \end{refsection}
}
{
  \hspace{3em}
  \parbox{10cm}{\getauthors}
  \vfill
}
%</mpi-aebr|mpi-far|mpi-plenary>
%    \end{macrocode}
%
% This is followed by the reports ISSN and ISBN numbers.
%
%    \begin{macrocode}
%<*mpi-aebr|mpi-far|mpi-plenary>

ISSN \@issn{} (online)\\
ISBN \@isbn{} (online)
\vfill
%</mpi-aebr|mpi-far|mpi-plenary>
%    \end{macrocode}
%
% Finally the report text block (and page) finishes with the date (month year).
%    \begin{macrocode}
%<*mpi-aebr|mpi-far|mpi-plenary>
\textbf{\@month{} \@date}
}
\end{minipage}
\end{textblock*}
\mbox{}
\clearpage
%</mpi-aebr|mpi-far|mpi-plenary>
%    \end{macrocode}
%
% The inside page contains a standard set of text.
%
%    \begin{macrocode}
%<*mpi-aebr|mpi-far|mpi-plenary>
{
\sffamily
\textbf{Disclaimer}

\small This document is published by Fisheries New Zealand, a business unit of the Ministry for Primary \\
Industries (MPI). The information in this publication is not government policy. While every effort has \\
been made to ensure the information is accurate, the Ministry for Primary Industries does not accept \\
any responsibility or liability for error of fact, omission, interpretation, or opinion that may be present, \\
nor for the consequence of any decisions based on this information. Any view or opinion expressed \\
does not necessarily represent the view of Fisheries New Zealand or the Ministry for Primary \\
Industries.

\vspace{10pt}
Requests for further copies should be directed to:

Fisheries Science Editor\\
Fisheries New Zealand\\
Ministry for Primary Industries\\
PO Box 2526\\
Wellington 6140\\
NEW ZEALAND

Email: Fisheries-Science.Editor@mpi.govt.nz\\
Telephone: 0800 00 83 33

This publication is also available on the Ministry for Primary Industries websites at:\\
\href{http://www.mpi.govt.nz/news-and-resources/publications}{http://www.mpi.govt.nz/news-and-resources/publications}\\
\href{http://fs.fish.govt.nz}{http://fs.fish.govt.nz} go to Document library/Research reports

\vspace{10pt}
\textbf{\copyright{} Crown Copyright -- Fisheries New Zealand}

\vspace{30pt}
Please cite this report as:

\begin{refsection}[\jobname-self.bib]
\renewcommand*{\multinamedelim}{\addsemicolon\space}
\renewcommand*{\finalnamedelim}{\addsemicolon\space}%
\renewcommand*{\bibinitdelim}{}
\AtNextCite{\defcounter{maxnames}{99}}
\citename{this}{author}\ (\citefield{this}{year}).\ \citefield{this}{title}. \emph{\citefield{this}{journaltitle}}. \pageref{LastPage}\ p.
\end{refsection}

}
\clearpage
%</mpi-aebr|mpi-far|mpi-plenary>
%    \end{macrocode}
%
% Finally we turn the header and footer off. These get turned on by the summary command.
%
%    \begin{macrocode}
%<*mpi-aebr|mpi-far|mpi-plenary>
\pagestyle{plain}
\addtocounter{page}{-2}
\pagestyle{empty}
\ifdraft{
\linenumbers
\doublespacing
}
}
%</mpi-aebr|mpi-far|mpi-plenary>
%    \end{macrocode}
%
% \section{Table of Contents}
%
% Reports requires a table of contents that is formatted in a very specific way.
%
%    \begin{macrocode}
%<mpi-aebr|mpi-far|mpi-plenary>\RequirePackage{tocloft}
%    \end{macrocode}
%
%
% Section heading are always shown in capitals
% in the contents page. This involves rebinding contents line, so that when the
% title is used to construct a hyperref url, it doesn't break the url. This is
% used in report and AEBR.
%
%    \begin{macrocode}
%<*mpi-aebr|mpi-far|mpi-plenary>
\let\dfly@contentsline=\contentsline
\renewcommand\contentsline[4]{\dfly@contentsline{#1}{\ifstrequal{#1}{section}{\MakeUppercase{#2}}{#2}}{#3}{#4}}
%</mpi-aebr|mpi-far|mpi-plenary>
%    \end{macrocode}
%
% The table of contents should have it's own page.  It should be called the TABLE OF CONTENTS.
% The Plenary is formatted as if it is printed and so clears a double page.
% From January 2024, AEBRs and FARs require a Plain language summary between the TOC and Exec. Summary
%
%    \begin{macrocode}
%<*mpi-plenary>
\RequirePackage{tocloft}
\renewcommand{\cfttoctitlefont}{\bfseries\sffamily}
\addto\captionsenglish{%
  \renewcommand{\contentsname}{TABLE OF CONTENTS}
}
\g@addto@macro\@cfttocfinish{\cleardoublepage}
\tocloftpagestyle{empty}
%</mpi-plenary>
%<*mpi-aebr|mpi-far>
\RequirePackage{tocloft}
\renewcommand{\cfttoctitlefont}{\bfseries\sffamily}
\addto\captionsenglish{%
  \renewcommand{\contentsname}{TABLE OF CONTENTS}
}
\g@addto@macro\@cfttocfinish{\clearpage\sffamily\textbf{Plain language summary}

\rmfamily {\@plainlanguagesummary} \cleardoublepage}
\tocloftpagestyle{empty}
%</mpi-aebr|mpi-far>
%    \end{macrocode}
%
% Section lines should be sans serif.
%    \begin{macrocode}
%<*mpi-aebr|mpi-far|mpi-plenary>
\g@addto@macro\cftsecpagefont{\sffamily}
\g@addto@macro\cftsecfont{\sffamily}
%</mpi-aebr|mpi-far|mpi-plenary>
%    \end{macrocode}
%
% \section{Floats}
%
%
% Captions are all bold.
%    \begin{macrocode}
%<mpi-aebr|mpi-far|mpi-plenary>\usepackage[font={small,bf},labelsep=colon,singlelinecheck=false]{caption}
%    \end{macrocode}
%
% We adjust the fptop and fpbot lengths to so that floats which are on a page on their own are
% pushed to the top.
%
%    \begin{macrocode}
%<*common>
\makeatletter
\setlength{\@fptop}{0pt}
\setlength{\@fpbot}{0pt plus 1fil}
\makeatother
%</common>
%    \end{macrocode}
%
% \section{Draft mode}
%
% Draft mode should create a watermark across every page to make them as draft.
% It should also turn on double spacing and some line numbering for ease of editing.
% For clarity line numbers and double spacing will only start \emph{after} the title
% page.
%    \begin{macrocode}
%<*common>
\RequirePackage{setspace}
\RequirePackage{ifdraft}
\ifdraft{
\RequirePackage{draftwatermark}
\usepackage[modulo]{lineno}
\modulolinenumbers[2]
\SetWatermarkColor{black!15}
\SetWatermarkFontSize{62pt}
\SetWatermarkText{\sffamily DRAFT - Not to be quoted}
}{}
%</common>
%    \end{macrocode}
%
% \section{Full landscape pages}
%
% Fisheries New Zealand require that footers are rotated on landscape pages.
% This is achieved by defining a new full landscape environment (flandscape).
% See https://github.com/dragonfly-science/mpi-latex-templates/issues/26
%    \begin{macrocode}
%<*mpi-aebr|mpi-far|mpi-plenary>
\RequirePackage{pdflscape}
\fancypagestyle{lscapefoot}{%
\fancyhf{} % clear all header and footer fields
\fancyfoot[LE]{%
\begin{textblock*}{1.0\textheight}[0,1](190mm,271mm){\rotatebox{90}{\rule{247mm}{0.4pt}}}\end{textblock*}
\begin{textblock*}{0.7\textwidth}[0,1](192mm,271mm){\rotatebox{90}{\sffamily\scriptsize\thepage\ {\normalsize$\bullet$}\ \footertitle}}\end{textblock*}
\begin{textblock*}{0.3\textwidth}(192mm,24mm){\rotatebox{90}{\sffamily\scriptsize Fisheries New Zealand}}\end{textblock*}}
\fancyfoot[LO]{%
\begin{textblock*}{1.0\textheight}[0,1](190mm,271mm){\rotatebox{90}{\rule{247mm}{0.4pt}}}\end{textblock*}
\begin{textblock*}{0.3\textwidth}[0,1](192mm,271mm){\rotatebox{90}{\sffamily\scriptsize Fisheries New Zealand}}\end{textblock*}
\begin{textblock*}{0.7\textwidth}(192mm,24mm){\rotatebox{90}{\sffamily\scriptsize\footertitle\ {\normalsize$\bullet$}\ \thepage}}\end{textblock*}}
\renewcommand{\headrulewidth}{0pt}
\renewcommand{\footrulewidth}{0pt}}
\newenvironment{flandscape}
    {\newgeometry{hmargin=2.5cm,vmargin=2.5cm}
     \thispagestyle{lscapefoot}
     \pagestyle{lscapefoot}
     \begin{landscape}
    }
    {\end{landscape}
     \restoregeometry
     \pagestyle{plain}
    }
%</mpi-aebr|mpi-far|mpi-plenary>
%    \end{macrocode}
%
%
%  \section{Additional Packages}
% We also various table-related packages, guided by the kableExtra requirements, 
% and set up the booktabs rules so kable tables conform to FNZ requirements.
%
%    \begin{macrocode}
%<*common>
\RequirePackage{booktabs}
\setlength\heavyrulewidth{0ex}
\setlength\lightrulewidth{0ex}
\RequirePackage{longtable}
\RequirePackage{float}
\RequirePackage{makecell}
\RequirePackage{xcolor}
\RequirePackage{colortbl}
\RequirePackage{multirow}
\RequirePackage[all]{nowidow}
%</common>
%    \end{macrocode}
%
% The SIunitx package is helpful for numbers and units.
% FNZ specify no space between numbers and a percent symbol
%    \begin{macrocode}
%<*common>
\RequirePackage{siunitx}
\sisetup{detect-all}
\DeclareSIUnit[quantity-product = ]\percent{\char`\%}
%</common>
%    \end{macrocode}
%
%    Hyperref must be the last package loaded. 
%    Links should not be coloured or have boxes around them.
%    \begin{macrocode}
%<*common>
\RequirePackage{hyperref}
\hypersetup{
  colorlinks=true,
  urlcolor=black,
  citecolor=black,
  anchorcolor=black,
  menucolor=black,
  linkcolor=black
}
%</common>
%    \end{macrocode}
%
% \section{Self Citing}
%
% In order to automatically generate the citation using biblatex, we
% write out a generated bib file that contains an entry for the current document.
% This must happen at the end of the prelude, so that the title and so on are defined.
% Note that new lines have to be suppressed while doing this, and |\emph|, |\nobreakspace|,
% |\newaffiliation|, |\useaffiliation| and |\addrsep| commands
% need to print out the command rather than actually be evaluated. As such, these
% are the only LaTeX commands supported in the author/title.
%
%    \begin{macrocode}
%<*mpi-aebr|mpi-far|mpi-plenary>
\RequirePackage{lastpage}
\newcommand{\newaffiliation}[2]{}
\newcommand{\useaffiliation}[1]{}
\newcommand{\addrsep}{}
\AtBeginDocument{
{
\renewcommand{\and}{ and }
\renewcommand{\nobreakspace}{\@backslashchar nobreakspace}
\renewcommand{\emph}[1]{\@backslashchar emph\@charlb #1\@charrb}
\renewcommand{\newaffiliation}[2]{\@backslashchar newaffiliation\@charlb #1\@charrb\@charlb #2\@charrb}
\renewcommand{\useaffiliation}[1]{\@backslashchar useaffiliation\@charlb #1\@charrb}
\renewcommand{\addrsep}{\@backslashchar addrsep}
\renewcommand{\\}{}
\edef\text{@article\@charlb this,^^J%
options = \@charlb skipbib \@charrb,^^J%
year = \@charlb \@date \@charrb,^^J%
title = \@charlb \@title \@charrb,^^J%
author = \@charlb \@author \@charrb,^^J%
journal = \@charlb {\@reportseries} \@reportnopre \@reportno \@charrb,^^J%
\@charrb
}
\newwrite\tempfile
\immediate\openout\tempfile=\jobname-self.bib
\immediate\write\tempfile{\text}
\immediate\closeout\tempfile
}
}
%</mpi-aebr|mpi-far|mpi-plenary>
%    \end{macrocode}
%
%
%    There is space for an executive summary, or some other
% unnumbered starting section.
%
%    \begin{macrocode}
%<*mpi-aebr|mpi-far|mpi-plenary>
 \NewDocumentCommand{\summary}{O {Executive Summary}}{
  \pagestyle{plain}
  \setcounter{page}{1}
  \phantomsection
  \addcontentsline{toc}{section}{#1}
  \section*{#1}
}
%</mpi-aebr|mpi-far|mpi-plenary>
%    \end{macrocode}
%
%
% We also need a facility to cite the paper that is currently being written.
% This needs to include author affiliations as footnotes and so requires a
% specific name format and modified apaauthor bibmacro
%    \begin{macrocode}
%<*mpi-aebr|mpi-far>
\usepackage{fnpct}
\DeclareNameFormat{affiliation}{%
  \ifuniquenametemplatename{affiliation}
    {\usebibmacro{name:affiliation:family-given}%
       {\namepartfamily}%
       {\namepartgiven}%
       {\namepartgiveni}%
       {\namepartprefix}%
       {\namepartsuffix}%
       {\namepartaffiliation}}
    {\usebibmacro{name:apa:family-given}%
       {\namepartfamily}%
       {\namepartgiven}%
       {\namepartgiveni}%
       {\namepartprefix}%
       {\namepartsuffix}}%
  \usebibmacro{name:andothers}}
\DeclareUniquenameTemplate[affiliation]{
   \namepart{family}%
   \namepart{given}%
   \namepart{giveni}%
   \namepart{prefix}%
   \namepart{suffix}%
   \namepart{affiliation}}

% Modified from name:apa:family-given
\newbibmacro*{name:affiliation:family-given}[6]{%
  \ifuseprefix
    {\usebibmacro{name:delim}{#4#1}%
     \usebibmacro{name:hook}{#4#1}%
     \ifdefvoid{#4}{}{%
       \mkbibnameprefix{#4}\isdot%
       \ifprefchar{}{\bibnamedelimc}}%
     \mkbibnamefamily{#1}\isdot%
     \ifdefvoid{#2}{}{\revsdnamepunct\bibnamedelimd\mkbibnamegiven{#3}\isdot%
                      \ifthenelse{\value{uniquename}>1}
                        {\bibnamedelimd\mkbibbrackets{#2}}
                        {}}%
     \ifdefvoid{#5}{}{\addcomma\bibnamedelimd\mkbibnamesuffix{#5}\isdot}%
     \ifdefvoid{#6}{}{#6}}
    {\usebibmacro{name:delim}{#1}%
     \usebibmacro{name:hook}{#1}%
     \mkbibnamefamily{#1}\isdot
     \ifboolexpe{%
       test {\ifdefvoid{#2}}
       and
       test {\ifdefvoid{#4}}}
       {}
       {\revsdnamepunct}%
     \ifdefvoid{#2}{}{\bibnamedelimd\mkbibnamegiven{#3}%
                      \ifthenelse{\value{uniquename}>1}
                        {\bibnamedelimd\mkbibbrackets{#2}}
                        {}}%
     \ifdefvoid{#4}{}{%
       \bibnamedelimc\mkbibnameprefix{#4}%
       \ifprefchar{}{\bibnamedelimc}}%
     \ifdefvoid{#5}{}{\addcomma\bibnamedelimd\mkbibnamesuffix{#5}\isdot}%
     \ifdefvoid{#6}{}{#6}}}

% The |\citeself| command inserts the document citation in the MPI-specified format
\newcommand{\citeself}{
  \renewcommand{\newaffiliation}[2]{
    \renewcommand{\addrsep}{, }
    \ifhmode\unskip\fi\footnote{##2}
    \newcounter{##1}
    \setcounter{##1}{\value{footnote}}
  }
  \renewcommand{\useaffiliation}[1]{
    \ifhmode\unskip\fi\footnotemark[\value{##1}]
  }
  \begin{refsection}[\jobname-self.bib]
  \renewcommand*{\multinamedelim}{\addsemicolon\space}
  \renewcommand*{\finalnamedelim}{\addsemicolon\space}%
  \renewcommand*{\bibinitdelim}{}
  \AtNextCite{\defcounter{maxnames}{99}}
  \vspace{11pt}
  \textbf{\citename{this}[affiliation]{author}\@(\citefield{this}{year}).\ \citefield{this}{title}.}

  \textbf{\emph{\citefield{this}{journaltitle}}. \pageref{LastPage}\ p.}
  \end{refsection}
}
%</mpi-aebr|mpi-far>
%    \end{macrocode}
%
% \section{Appendices}
%
%
% We require appendices to be printed as ``Appendix X'' in the table of
% contents. This is accomplished by using the appendix package with the
% titletoc option.
%    \begin{macrocode}
%<*mpi-aebr|mpi-far|mpi-plenary>
\RequirePackage[titletoc]{appendix}
%</mpi-aebr|mpi-far|mpi-plenary>
%    \end{macrocode}
%
% In the appendix we also wish to customize a number of properties. To do this
% we simply add these to the |\appendices| command.  These are:
% \begin{itemize}
%\item Section headings printed as Appendix X
%\item Tables, equations and figures get the appendix letter before their number
%\item Counters are reset to set
% \end{itemize}
%    \begin{macrocode}
%<*mpi-aebr|mpi-far|mpi-plenary>
\RequirePackage{chngcntr}
\g@addto@macro\appendices{
\titleformat{\section}{\normalfont\bfseries\sffamily\uppercase}{}{0em}{APPENDIX \thesection: }
\setcounter{section}{0}  % reset counter
\counterwithin{figure}{section} % Figures numbered within appendix
\counterwithin{table}{section} % Figures numbered within appendix
\counterwithin{equation}{section} % Figures numbered within appendix
\renewcommand{\theequation}{\thesection-\arabic{equation}}
\renewcommand{\thetable}{\thesection-\arabic{table}}
\renewcommand{\thefigure}{\thesection-\arabic{figure}}
% Restore citation delimiters that were changed for the bibliography
\renewcommand*{\multicitedelim}{,\addspace}
\renewcommand*{\multinamedelim}{\addcomma\addspace}
\renewcommand*{\finalnamedelim}{\addspace\&\addspace}
}
%</mpi-aebr|mpi-far|mpi-plenary>
%    \end{macrocode}
%
% When using |\autoref| we want the appendix number to be prefixed with the
% word Appendix:
%    \begin{macrocode}
%<*mpi-aebr|mpi-far|mpi-plenary>
\newcommand*{\Appendixautorefname}{Appendix}
%</mpi-aebr|mpi-far|mpi-plenary>
%    \end{macrocode}

% \section{Integration}
%
% We overwrite commands used in Dragonfly report templates, so that we are able to reformat documents simply by changing
% the document class
%    \begin{macrocode}
%<*common>
\newcommand{\middleimage}[2]{}
\newcommand{\rightimage}[2]{}
\newcommand{\leftimage}[2]{}
\newcommand{\singleimage}[2]{}
\newcommand{\subtitle}[1]{}
\newcommand{\titleimage}[1]{}
\newcommand{\licence}[1]{}
%</common>
%    \end{macrocode}

\endinput
