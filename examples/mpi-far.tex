% This is the document class for FARs, derived from the latex article
% class

\documentclass{mpi-far}

%If a document is  typeset as draft it will be double spaced, with
%linenumbers, and with a watermark \documentclass[draft]{aebr}

% The FAR requires two titles. The first is the short title which
% goes on the page footers. The second is the long title which goes on
% the cover page.  You are only allowed to use the LaTeX macros \\
% and \emph inside of titles safely. Others may result in
% unpredictable behaviour. 

\title{Example report}{An example Fisheries Assessment Report (FAR)}

% Each FAR has a subtitle
\subtitle{A report for testing purposes}

%The date must be split into two parts for compatibility with reports. 
% The year is specified with \date, and the month is specified with \reportmonth
\date{2015}
\reportmonth{January}

% Authors must be listed with their full names separated by \and.
\author{John Smith \and Jane E. Smith}

% The subcaption package is used to create subfigures in this example.
\usepackage{subcaption}

% The lipsum package is used to generate filler text. This won't be
% needed in your actual report.
\usepackage{lipsum}

% Each report has an ISBN and report number. If any of these are not 
% included a place holder value will be used.
\isbn{XX-XXXXX-XX}
\reportno{XX}

% This specified the location of the bibliography file
\addbibresource{test.bib}

% The document begins here
\begin{document}

% Generate the title page
\maketitle

% Generate the table of contents.
\tableofcontents

% An FAR must contain an executive summary. Which \summary will generate a 
% title and contents entry for. 

\summary

% The summary must start with a citation for the current document which can be 
% automatically generated using \citeself.
\citeself

\lipsum[1]
\clearpage


\section{Introduction}

%filler text
\lipsum[1]


\begin{figure}[h]
  \begin{center}  
  \includegraphics[width=40mm,height=40mm]{FAR}
  \end{center}  
  \caption{This is a caption that is longer than one line. We are just
  making sure that it wraps successfully. The figure is the cover of
  the report.}
\end{figure}

\section{Methods}

\subsection{Details}

\lipsum[2]

% An example of how to do subfigures using subcaption.
\begin{figure}[h]
\begin{center}
  \begin{subfigure}{0.45\textwidth}
  \includegraphics[width=40mm]{FAR}
  \caption{Cover of a Fisheries Assessment Report (FAR).}
\end{subfigure}\qquad
  \begin{subfigure}{0.45\textwidth}
  \includegraphics[width=40mm]{FAR}
  \caption{Cover of an Aquatic Environment and Biodiversity Report
  (AEBR).}
\end{subfigure}
\end{center}
\end{figure}


\section{Results}
\section{Discussion}

\subsection{A first level subsection}
\subsubsection{A second level subsection}
\lipsum[3] 
\paragraph{A minor heading}
\lipsum[5] 

Examples of how citations can be used in different ways

% Examples of how to do citations. Note the cite, citet and citep work as standard. 
\begin{itemize}
  \item citet \citet{baker_nzclassification_2010}
  \item citep \citep{doc_sealion_2009}
  \item parencite \parencite{gales_phocarctos_2008}
  \item nptextcite \nptextcite{ mpi_review_2012}
  \item fullcite \fullcite{mpi_review_2012}
  \item fullcitebib \fullcitebib{mpi_review_2012}
  \item citeyear \citeyear{robertson_population_2011}
  %\item \footcite{mpi_review_2012}
  \item textcite \textcite{roe_necropsy_2007}
\end{itemize}

% Bibliography should be printed starting on a new page.
\clearpage
\printbibliography

%End of document
\end{document}
